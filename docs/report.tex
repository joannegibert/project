\documentclass{article}
\usepackage[margin=2.5cm]{geometry}
\title{SEIRD Epidemiological Model}
\author{Zoé Jacob & Joanne Gibert}
\date{December 5, 2025}
\begin{document}
\maketitle

Poultry farming has become increasingly important to meet the demands of a growing population, which has led to the expansion of intensive farming practices. However, in high-density farms, where chicken are kept in close quarters, certain diseases spread more easily. One such disease is Highly Pathogenic Avian Influenza (HPAI), which can devastate an avian population, killing between 75\% and 100\% of individuals. 

CONSIGNES DU PROF (The final report should be approximately 5 pages without references and
authorship statement, but this is not a hard limit. However, if many
figures are generated, they can be included in the appendix to keep the
main report centered around the main conclusions.

With exception to Point 1, the final report should be a standalone
document that can be read independently of the project proposal (i.e.,
your client should be able to understand the project without having to
refer back to the original proposal). Therefore, you may reuse (i.e.,
copy) parts of your project proposal here. The report should be written
in a professional manner suitable for submission to a client or
publication.) CONSIGNES DU PROF 

\section{Deviations from project
proposal}\label{deviations-from-project-proposal}
In the project proposal, we decided to consider six scenarios by varying the transmission rate $\beta$ and the recovery rate $\gamma$. Finally, we will simulate nine scenarios by also varying the mortality rate $\mu$. 

CONSIGNES DU PROF (Describe briefly major deviations from the original project proposal, if
any, in terms of topic, scope, approach, or schedule.) CONSIGNES DU PROF

\section{Introduction to the problem}\label{introduction-to-the-problem}

Why would anyone be interested in this topic: curiosity, societal
implications, development of the technology? Provide citation to
literature where appropriate.

Definition of project scope:

\begin{itemize}
\tightlist
\item
  what processes will you include / exclude
\item
  what scenarios will you consider
\end{itemize}

\emph{Note that novelty of the question, approach, or finding is not a
requirement for this project.}

\section{Approach used}\label{approach-used}

Describe the approach taken to solve the problem. Include relevant
mathematical relationships, models, algorithms, data, etc. Is the model
mechanistic or empirical (e.g., conservation equation, or a parametrized
relationship between input and output)? Do you use the program for
forecasting/prediction, or inference (e.g., understand model
parameters)?

Provide citation to literature where appropriate, particularly to
compare your approach to existing work (whether it is similar or
different).

\section{Results}\label{results}

Describe the results. Give your assessment of whether they are
reasonable - and how do you determine this?

\section{Conclusion and outlook}\label{conclusion-and-outlook}

Summarize the approach taken and the answer to the question set out in
the problem statement. Describe limitations of the work (outlook) and
how it could be improved.

\section{Authorship statement}\label{authorship-statement}

Describe contributions of each student on the team to the project in
terms of writing of code and report, or project management.

\section{References}\label{references}

List all references cited in the report in a consistent format. Every
reference should have a citation in the text, and every citation in the
text should have a corresponding reference.

\end{document}


